% Options for packages loaded elsewhere
\PassOptionsToPackage{unicode}{hyperref}
\PassOptionsToPackage{hyphens}{url}
%
\documentclass[
]{article}
\usepackage{amsmath,amssymb}
\usepackage{iftex}
\ifPDFTeX
  \usepackage[T1]{fontenc}
  \usepackage[utf8]{inputenc}
  \usepackage{textcomp} % provide euro and other symbols
\else % if luatex or xetex
  \usepackage{unicode-math} % this also loads fontspec
  \defaultfontfeatures{Scale=MatchLowercase}
  \defaultfontfeatures[\rmfamily]{Ligatures=TeX,Scale=1}
\fi
\usepackage{lmodern}
\ifPDFTeX\else
  % xetex/luatex font selection
\fi
% Use upquote if available, for straight quotes in verbatim environments
\IfFileExists{upquote.sty}{\usepackage{upquote}}{}
\IfFileExists{microtype.sty}{% use microtype if available
  \usepackage[]{microtype}
  \UseMicrotypeSet[protrusion]{basicmath} % disable protrusion for tt fonts
}{}
\makeatletter
\@ifundefined{KOMAClassName}{% if non-KOMA class
  \IfFileExists{parskip.sty}{%
    \usepackage{parskip}
  }{% else
    \setlength{\parindent}{0pt}
    \setlength{\parskip}{6pt plus 2pt minus 1pt}}
}{% if KOMA class
  \KOMAoptions{parskip=half}}
\makeatother
\usepackage{xcolor}
\usepackage[margin=1in]{geometry}
\usepackage{color}
\usepackage{fancyvrb}
\newcommand{\VerbBar}{|}
\newcommand{\VERB}{\Verb[commandchars=\\\{\}]}
\DefineVerbatimEnvironment{Highlighting}{Verbatim}{commandchars=\\\{\}}
% Add ',fontsize=\small' for more characters per line
\usepackage{framed}
\definecolor{shadecolor}{RGB}{248,248,248}
\newenvironment{Shaded}{\begin{snugshade}}{\end{snugshade}}
\newcommand{\AlertTok}[1]{\textcolor[rgb]{0.94,0.16,0.16}{#1}}
\newcommand{\AnnotationTok}[1]{\textcolor[rgb]{0.56,0.35,0.01}{\textbf{\textit{#1}}}}
\newcommand{\AttributeTok}[1]{\textcolor[rgb]{0.13,0.29,0.53}{#1}}
\newcommand{\BaseNTok}[1]{\textcolor[rgb]{0.00,0.00,0.81}{#1}}
\newcommand{\BuiltInTok}[1]{#1}
\newcommand{\CharTok}[1]{\textcolor[rgb]{0.31,0.60,0.02}{#1}}
\newcommand{\CommentTok}[1]{\textcolor[rgb]{0.56,0.35,0.01}{\textit{#1}}}
\newcommand{\CommentVarTok}[1]{\textcolor[rgb]{0.56,0.35,0.01}{\textbf{\textit{#1}}}}
\newcommand{\ConstantTok}[1]{\textcolor[rgb]{0.56,0.35,0.01}{#1}}
\newcommand{\ControlFlowTok}[1]{\textcolor[rgb]{0.13,0.29,0.53}{\textbf{#1}}}
\newcommand{\DataTypeTok}[1]{\textcolor[rgb]{0.13,0.29,0.53}{#1}}
\newcommand{\DecValTok}[1]{\textcolor[rgb]{0.00,0.00,0.81}{#1}}
\newcommand{\DocumentationTok}[1]{\textcolor[rgb]{0.56,0.35,0.01}{\textbf{\textit{#1}}}}
\newcommand{\ErrorTok}[1]{\textcolor[rgb]{0.64,0.00,0.00}{\textbf{#1}}}
\newcommand{\ExtensionTok}[1]{#1}
\newcommand{\FloatTok}[1]{\textcolor[rgb]{0.00,0.00,0.81}{#1}}
\newcommand{\FunctionTok}[1]{\textcolor[rgb]{0.13,0.29,0.53}{\textbf{#1}}}
\newcommand{\ImportTok}[1]{#1}
\newcommand{\InformationTok}[1]{\textcolor[rgb]{0.56,0.35,0.01}{\textbf{\textit{#1}}}}
\newcommand{\KeywordTok}[1]{\textcolor[rgb]{0.13,0.29,0.53}{\textbf{#1}}}
\newcommand{\NormalTok}[1]{#1}
\newcommand{\OperatorTok}[1]{\textcolor[rgb]{0.81,0.36,0.00}{\textbf{#1}}}
\newcommand{\OtherTok}[1]{\textcolor[rgb]{0.56,0.35,0.01}{#1}}
\newcommand{\PreprocessorTok}[1]{\textcolor[rgb]{0.56,0.35,0.01}{\textit{#1}}}
\newcommand{\RegionMarkerTok}[1]{#1}
\newcommand{\SpecialCharTok}[1]{\textcolor[rgb]{0.81,0.36,0.00}{\textbf{#1}}}
\newcommand{\SpecialStringTok}[1]{\textcolor[rgb]{0.31,0.60,0.02}{#1}}
\newcommand{\StringTok}[1]{\textcolor[rgb]{0.31,0.60,0.02}{#1}}
\newcommand{\VariableTok}[1]{\textcolor[rgb]{0.00,0.00,0.00}{#1}}
\newcommand{\VerbatimStringTok}[1]{\textcolor[rgb]{0.31,0.60,0.02}{#1}}
\newcommand{\WarningTok}[1]{\textcolor[rgb]{0.56,0.35,0.01}{\textbf{\textit{#1}}}}
\usepackage{graphicx}
\makeatletter
\def\maxwidth{\ifdim\Gin@nat@width>\linewidth\linewidth\else\Gin@nat@width\fi}
\def\maxheight{\ifdim\Gin@nat@height>\textheight\textheight\else\Gin@nat@height\fi}
\makeatother
% Scale images if necessary, so that they will not overflow the page
% margins by default, and it is still possible to overwrite the defaults
% using explicit options in \includegraphics[width, height, ...]{}
\setkeys{Gin}{width=\maxwidth,height=\maxheight,keepaspectratio}
% Set default figure placement to htbp
\makeatletter
\def\fps@figure{htbp}
\makeatother
\setlength{\emergencystretch}{3em} % prevent overfull lines
\providecommand{\tightlist}{%
  \setlength{\itemsep}{0pt}\setlength{\parskip}{0pt}}
\setcounter{secnumdepth}{-\maxdimen} % remove section numbering
\ifLuaTeX
  \usepackage{selnolig}  % disable illegal ligatures
\fi
\usepackage{bookmark}
\IfFileExists{xurl.sty}{\usepackage{xurl}}{} % add URL line breaks if available
\urlstyle{same}
\hypersetup{
  pdftitle={Simple Linear Regression (SLR)},
  pdfauthor={Davis Ogden},
  hidelinks,
  pdfcreator={LaTeX via pandoc}}

\title{Simple Linear Regression (SLR)}
\author{Davis Ogden}
\date{2024-06-16}

\begin{document}
\maketitle

\subsection{Simple Linear Regression (Homework
1)}\label{simple-linear-regression-homework-1}

\subsubsection{Part 1: Overview of
Dataset}\label{part-1-overview-of-dataset}

This homework assignment is based on a project on life expectancy
research.

\paragraph{Background}\label{background}

Life expectancy is a measure of the average lifespan of a person born
into a nation. We plan to examine the factors that can increase or
decrease life expectancy and their interactions. There are almost 200
nations in the world, each with unique living conditions, economic
status, and healthcare treatments. Our chosen data set contains data for
178 countries, over the 15-year period from 2000- 2015, as well as a
variety of health-related and economic data about the nation for each
year. It is important to note, some nations were excluded from the
dataset because of repeated missing values due to difficulty finding
data from smaller nations such as Togo, Vanuatu, and Cape Verde. The
dataset contains four main groupings of predictor variables
immunizations, mortalities, economic, and social. Data has been
collected from multiple resources: the national health and economic
dataset have been collected from Kaggle, and the number of medical
professionals (doctors, nurses, and pharmacists, per 10,000 people in
the nation) have been collected from the WHO.

\begin{Shaded}
\begin{Highlighting}[]
\CommentTok{\#read data into file life\_data data frame}
\NormalTok{life\_data }\OtherTok{=} \FunctionTok{data.frame}\NormalTok{(}\FunctionTok{read.csv}\NormalTok{(}\AttributeTok{file =} \StringTok{\textquotesingle{}life\_expectancy.csv\textquotesingle{}}\NormalTok{))}
\FunctionTok{head}\NormalTok{(life\_data)}
\end{Highlighting}
\end{Shaded}

\begin{verbatim}
##               Country X2015Life.expectancy Medical.doctors Nurses Pharmacists
## 1         Afghanistan                 65.0            2.54   4.52        0.29
## 2             Albania                 77.8           18.83  58.31       10.86
## 3             Algeria                 75.6           17.32  15.59        4.49
## 4              Angola                 52.4            2.11   4.01        0.74
## 5 Antigua and Barbuda                 76.4           28.98  95.77        1.83
## 6           Argentina                 76.3           38.95  54.47        4.97
\end{verbatim}

\subsubsection{Problem 1}\label{problem-1}

Estimate the parameters for a linear regression to predict 𝑌 based on 𝑋.
Complete the following with details.

\begin{Shaded}
\begin{Highlighting}[]
\CommentTok{\#1.a}
\NormalTok{avg\_doctor }\OtherTok{=} \FunctionTok{mean}\NormalTok{(life\_data}\SpecialCharTok{$}\NormalTok{Medical.doctors)}
\NormalTok{avg\_doctor}
\end{Highlighting}
\end{Shaded}

\begin{verbatim}
## [1] 19.44011
\end{verbatim}

\begin{Shaded}
\begin{Highlighting}[]
\CommentTok{\#1.b}
\NormalTok{avg\_life }\OtherTok{=} \FunctionTok{mean}\NormalTok{(life\_data}\SpecialCharTok{$}\NormalTok{X2015Life.expectancy)}
\NormalTok{avg\_life}
\end{Highlighting}
\end{Shaded}

\begin{verbatim}
## [1] 71.64157
\end{verbatim}

\begin{Shaded}
\begin{Highlighting}[]
\CommentTok{\#1.c {-} SSxy}
\NormalTok{x\_xbar }\OtherTok{=}\NormalTok{ life\_data}\SpecialCharTok{$}\NormalTok{Medical.doctors}\SpecialCharTok{{-}}\NormalTok{avg\_doctor}
\NormalTok{y\_ybar }\OtherTok{=}\NormalTok{ life\_data}\SpecialCharTok{$}\NormalTok{X2015Life.expectancy}\SpecialCharTok{{-}}\NormalTok{avg\_life}
\NormalTok{SSxy }\OtherTok{=} \FunctionTok{sum}\NormalTok{(x\_xbar}\SpecialCharTok{*}\NormalTok{y\_ybar)}
\NormalTok{SSxy}
\end{Highlighting}
\end{Shaded}

\begin{verbatim}
## [1] 18444.76
\end{verbatim}

\begin{Shaded}
\begin{Highlighting}[]
\CommentTok{\#1.d {-} SSxx}
\NormalTok{SSxx }\OtherTok{=} \FunctionTok{sum}\NormalTok{(x\_xbar}\SpecialCharTok{\^{}}\DecValTok{2}\NormalTok{)}
\NormalTok{SSxx}
\end{Highlighting}
\end{Shaded}

\begin{verbatim}
## [1] 52922.52
\end{verbatim}

\begin{Shaded}
\begin{Highlighting}[]
\CommentTok{\#1.e \& 1.f {-} b\_o \& b\_1}
\NormalTok{b\_1 }\OtherTok{=}\NormalTok{ SSxy}\SpecialCharTok{/}\NormalTok{SSxx}
\NormalTok{b\_o }\OtherTok{=}\NormalTok{ avg\_life }\SpecialCharTok{{-}}\NormalTok{ b\_1}\SpecialCharTok{*}\NormalTok{avg\_doctor}
\NormalTok{b\_1}
\end{Highlighting}
\end{Shaded}

\begin{verbatim}
## [1] 0.3485238
\end{verbatim}

\begin{Shaded}
\begin{Highlighting}[]
\NormalTok{b\_o}
\end{Highlighting}
\end{Shaded}

\begin{verbatim}
## [1] 64.86623
\end{verbatim}

\begin{Shaded}
\begin{Highlighting}[]
\CommentTok{\#1.g {-} SSE}
\NormalTok{life\_data}\SpecialCharTok{$}\NormalTok{Y\_hat\_life }\OtherTok{=}\NormalTok{ b\_o }\SpecialCharTok{+}\NormalTok{ b\_1}\SpecialCharTok{*}\NormalTok{life\_data}\SpecialCharTok{$}\NormalTok{Medical.doctors}
\NormalTok{SSE }\OtherTok{=} \FunctionTok{sum}\NormalTok{((life\_data}\SpecialCharTok{$}\NormalTok{X2015Life.expectancy}\SpecialCharTok{{-}}\NormalTok{life\_data}\SpecialCharTok{$}\NormalTok{Y\_hat\_life)}\SpecialCharTok{\^{}}\DecValTok{2}\NormalTok{)}

\CommentTok{\#1.h {-} MSE}
\NormalTok{MSE }\OtherTok{=}\NormalTok{ SSE}\SpecialCharTok{/}\NormalTok{(}\FunctionTok{nrow}\NormalTok{(life\_data)}\SpecialCharTok{{-}}\DecValTok{2}\NormalTok{)}

\CommentTok{\#1.i {-} SST}
\NormalTok{SST }\OtherTok{=} \FunctionTok{sum}\NormalTok{((life\_data}\SpecialCharTok{$}\NormalTok{X2015Life.expectancy}\SpecialCharTok{{-}}\NormalTok{avg\_life)}\SpecialCharTok{\^{}}\DecValTok{2}\NormalTok{)}

\CommentTok{\#1.j {-} Verify}
\NormalTok{SSR }\OtherTok{=} \FunctionTok{sum}\NormalTok{((avg\_life}\SpecialCharTok{{-}}\NormalTok{life\_data}\SpecialCharTok{$}\NormalTok{Y\_hat\_life)}\SpecialCharTok{\^{}}\DecValTok{2}\NormalTok{)}
\NormalTok{check }\OtherTok{=}\NormalTok{ SSR }\SpecialCharTok{+}\NormalTok{ SSE}
\FunctionTok{round}\NormalTok{(SST) }\SpecialCharTok{==} \FunctionTok{round}\NormalTok{(check)}
\end{Highlighting}
\end{Shaded}

\begin{verbatim}
## [1] TRUE
\end{verbatim}

\subsubsection{Problem 2}\label{problem-2}

In order to estimate the linear impact of 𝑋 on 𝑌, at a confidence of
(1−𝛼)\%, you should use the critical value, or the t value denoted as t(
\(1-\alpha/2\) , \(n-2\)), which has a value of \(1.653\) (use basic R
function or Excel for the exact value), at 𝛼=0.1, and \(1.973\) at
𝛼=0.05. The standard error of the estimation
\(s\{b1\} = \sqrt(\frac{MSE}{\Sigma(X_{i}-\bar{X})^2}=0.024\) . The
margin error, or 𝑡∗𝑆𝐸, of the confidence interval is \(0.039\) at 𝛼=0.1,
and \(0.047\) at 𝛼=0.05.

\begin{Shaded}
\begin{Highlighting}[]
\NormalTok{alpha }\OtherTok{=} \FloatTok{0.1}
\NormalTok{n }\OtherTok{=} \FunctionTok{nrow}\NormalTok{(life\_data)}
\NormalTok{t\_0}\FloatTok{.1} \OtherTok{=} \FunctionTok{qt}\NormalTok{(}\DecValTok{1}\FloatTok{{-}0.5}\SpecialCharTok{*}\NormalTok{alpha,n}\DecValTok{{-}2}\NormalTok{)}
\NormalTok{t\_0}\FloatTok{.1}
\end{Highlighting}
\end{Shaded}

\begin{verbatim}
## [1] 1.653557
\end{verbatim}

\begin{Shaded}
\begin{Highlighting}[]
\NormalTok{alpha }\OtherTok{=} \FloatTok{0.05}
\NormalTok{t\_0}\FloatTok{.05} \OtherTok{=} \FunctionTok{qt}\NormalTok{(}\DecValTok{1}\FloatTok{{-}0.5}\SpecialCharTok{*}\NormalTok{alpha,n}\DecValTok{{-}2}\NormalTok{)}
\NormalTok{t\_0}\FloatTok{.05}
\end{Highlighting}
\end{Shaded}

\begin{verbatim}
## [1] 1.973534
\end{verbatim}

\begin{Shaded}
\begin{Highlighting}[]
\NormalTok{SE\_b1 }\OtherTok{=} \FunctionTok{sqrt}\NormalTok{(MSE}\SpecialCharTok{/}\NormalTok{(}\FunctionTok{sum}\NormalTok{((life\_data}\SpecialCharTok{$}\NormalTok{Medical.doctors}\SpecialCharTok{{-}}\NormalTok{avg\_doctor)}\SpecialCharTok{\^{}}\DecValTok{2}\NormalTok{)))}
\NormalTok{SE\_b1}
\end{Highlighting}
\end{Shaded}

\begin{verbatim}
## [1] 0.02367142
\end{verbatim}

\begin{Shaded}
\begin{Highlighting}[]
\NormalTok{t\_0}\FloatTok{.1}\SpecialCharTok{*}\NormalTok{SE\_b1}
\end{Highlighting}
\end{Shaded}

\begin{verbatim}
## [1] 0.03914206
\end{verbatim}

\begin{Shaded}
\begin{Highlighting}[]
\NormalTok{t\_0}\FloatTok{.05}\SpecialCharTok{*}\NormalTok{SE\_b1}
\end{Highlighting}
\end{Shaded}

\begin{verbatim}
## [1] 0.04671637
\end{verbatim}

\subsubsection{Problem 3}\label{problem-3}

Perform a hypothesis test on the linear impact of 𝑋 on 𝑌, with a T test
with a significant value of 0.1.

\subsubsection{Problem 4}\label{problem-4}

Use R to obtain a summary of this SLR model. Highlight the following
concepts on the output, the notation, the values, and finally an
interpretation. Compute the item with R, or Excel if it is not directly
available in the R model summary output.

\begin{Shaded}
\begin{Highlighting}[]
\NormalTok{model }\OtherTok{=} \FunctionTok{lm}\NormalTok{(life\_data}\SpecialCharTok{$}\NormalTok{X2015Life.expectancy}\SpecialCharTok{\textasciitilde{}}\NormalTok{life\_data}\SpecialCharTok{$}\NormalTok{Medical.doctors)}
\FunctionTok{summary}\NormalTok{(model)}
\end{Highlighting}
\end{Shaded}

\begin{verbatim}
## 
## Call:
## lm(formula = life_data$X2015Life.expectancy ~ life_data$Medical.doctors)
## 
## Residuals:
##      Min       1Q   Median       3Q      Max 
## -14.1102  -3.5062   0.4287   4.0203  11.7057 
## 
## Coefficients:
##                           Estimate Std. Error t value Pr(>|t|)    
## (Intercept)               64.86623    0.61511  105.45   <2e-16 ***
## life_data$Medical.doctors  0.34852    0.02367   14.72   <2e-16 ***
## ---
## Signif. codes:  0 '***' 0.001 '**' 0.01 '*' 0.05 '.' 0.1 ' ' 1
## 
## Residual standard error: 5.446 on 176 degrees of freedom
## Multiple R-squared:  0.5519, Adjusted R-squared:  0.5494 
## F-statistic: 216.8 on 1 and 176 DF,  p-value: < 2.2e-16
\end{verbatim}

a). \(0.34852\) b). \(5.446\) c). \(176\) d).

\subsubsection{Problem 5}\label{problem-5}

\end{document}
